\documentclass[%
  a4paper,%
  oneside,
  nexus,
  11pt,% <10pt, 9pt>
  %style=screen,
  %sender=bottom,
  blue,% <orange, green, violet>
  %rgb, <cmyk>
  %mono,
  %extramargin,
  %marginleft, <marginright>
  ]{tubsreprt}

\usepackage[utf8x]{inputenc}
\usepackage[ngerman]{babel}
\usepackage[numbers]{natbib}
\usepackage{graphicx}
\usepackage{amsmath}                    % Packages für Formeln
\usepackage{amsthm}			% -||-
\usepackage{amsfonts}			% -||-
\usepackage{amssymb}			% -||-
\usepackage{lscape}			% Querformatiges Beschreiben einzelner Abschnitte mit \begin{landscape} Text \end{landscape}


% == Zeichen Tools und Gantt-Chart ==
%\usepackage{tikz} 			% TikZ - ein geniales Zeichentool vor allem für Skizzen oder Schemata, Programmabläufe, etc.
%\usepackage{pgf}			% Für Tikz
%\usepackage{pgfplots} 			% Plots mit TikZ
%\usetikzlibrary{shapes,backgrounds, arrows, positioning, trees, shadows, matrix, decorations.pathreplacing}	% Bibliotheken für Tikz
%\usepackage{pgfgantt}			% Zum Erstellen eines Gantt-Charts

\usepackage{longtable}
\usepackage{booktabs}
\usepackage{multirow}
\usepackage{verbatim}	% für mehrzeilige Kommentare \beginn{comment}

% Nomenklaturverzeichnis
\usepackage{setspace}
\usepackage{nomencl}
\usepackage{nomentbl}
\renewcommand{\nomname}{Nomenklaturverzeichnis}			% Änderung des Verzeichnistitels
\renewcommand{\nomAname}{Lateinische Formelzeichen}
\renewcommand{\nomGname}{Griechische Formelzeichen}
\renewcommand{\nomXname}{Abkürzungen}
\renewcommand{\nomZname}{Indizes}
\makenomenclature

\graphicspath{{./Bilder/}}

%===============================================================
%
%	Hyperlinks im Dokument
%
%===============================================================

\usepackage[
  pdftex,%
  hidelinks,%
  linktocpage, breaklinks
]{hyperref}				% Package für Lesezeichen und Verlinkungen

%===============================================================
%
%	Glossary
%
%===============================================================

\usepackage[nonumberlist,acronym,section]{glossaries}
	%nonumberlist: no page numbering
	%acronym: abbreviations list
	%toc: entry in table of contents
	%section: in toc placed at section level
	%nopostdot: no dots after description
				
\newglossary[slg1]{symbolslist}{syi}{syg}{List of Symbols}			% symbols
\newglossary[alg]{acronymlist}{acr}{acn}{List of Abbreviations}			% abbreviations	

\makeglossaries

\AtEndDocument{\glsaddall}	%füge alle Symbole dem Verzeichnis zu


%Makros für eine erleichterte Eingabe wiederkehrender Befehle:
\newcommand{\abb}[1]{Abbildung~\ref{#1}}
\newcommand{\tab}[1]{Tabelle~\ref{#1}}
\newcommand{\glg}[1]{Gleichung~\ref{#1}}
\newcommand{\kap}[1]{Kapitel~\ref{#1}}
\newcommand{\abschn}[1]{Abschnitt~\ref{#1}}
\newcommand{\anh}[1]{Anhang~\ref{#1}}

% Überschriften
\titleformat{\subsubsection}{\large\bfseries}{\subsubsection}{1em}{}
\newcommand{\subsubsubsection}[1]{\vspace{0.5cm} \noindent \textbf{{\normalsize #1}}\\}

\newcommand{\ul}{\underline}
\newcommand{\ol}{\overline}
\newcommand{\tb}{\textbf}
\newcommand{\ts}{\textsl}


%===============================================================
%
% Farben
%
%===============================================================

\definecolor{ilr-blue}{RGB}{44,79,162}
\definecolor{light-blue}{RGB}{220,220,255}
\definecolor{gray90}{gray}{0.9}
\definecolor{gray80}{gray}{0.8}
\definecolor{gray60}{gray}{0.6}
\definecolor{gray50}{gray}{0.5}
\definecolor{gray40}{gray}{0.4}
\definecolor{gray20}{gray}{0.2}

%===============================================================
%
% Titelseiten-Elemente
%
%===============================================================

%\title{\LARGE Laborbericht zum Flugversuch Do 128-6 \newline\newline }
%\subtitle{Institut für Flugführung}
%\author{\normalsize Marco De Geatano, Jens Karch, Philip Margenfeld, Kilian Schulz \& Wentao Wu}

\newcommand{\Art}{Laborbericht}											% Projekt, Bachelor, Studien, Master oder Diplom	
\newcommand{\Titel}{Flugversuch Do 128-6}								% Titel der Arbeit		
\newcommand{\Pruefer}{Mark Bitter (mark.bitter@tu-braunschweig.de)}		% Prüfer/Betreuer

%---------------------------------------------------------------

\logo{\includegraphics{ifflogo.png}}
\titlepicture{title_picture.jpg}

%===============================================================
%
% Hurenkinder/Schusterjungen vermeiden
%
%===============================================================
\clubpenalty         = 10000
\widowpenalty        = 10000
\displaywidowpenalty = 10000

%---------------------------------------------------------------------------------------------------

\begin{document}
	%\maketitle[image] %[<plain/image/imagetext>,<logo=left/right>] % TUBS Report Titelseite
	
	% === Titelseite === %
	\begin{titlepage}
 \centering

\begin{figure}[htbp]
 \begin{center}
  \vspace*{-4.0cm}
  \hspace*{-0.8cm}
  \includegraphics[width=1.15\textwidth]{banner_title.jpg} 
 \end{center}
\end{figure}
 
\begin{figure}[htbp]
 \begin{center}
  \hspace*{-0.8cm}
  \includegraphics[width=1.09\textwidth]{banner_do128.jpg}
 \end{center}
\end{figure}

 \vspace*{3.0cm}

 \textbf{\large {\Art}}


 \vspace*{1.2cm}
 
 \parbox[c]{\textwidth}{\centering \bfseries \LARGE \Titel}
 \vspace*{3cm}
 
\begin{tabular}{ll}
\textbf{\large Marco De Gaetano}  & Matr.Nr.: 4813585 \\
\textbf{\large Jens Karch}        & Matr.Nr.: 4790996 \\
\textbf{\large Philip Margenfeld} & Matr.Nr.: 4791034 \\
\textbf{\large Kilian Schultz}    & Matr.Nr.: 4510545 \\
\textbf{\large Wentao Wu}         & Matr.Nr.: 4787695
\end{tabular}

 \vspace*{2.5cm}

  \begin{center}
	Betreuer: \Pruefer
  \end{center}

\end{titlepage}

  
	\newpage
	
	% === Inhaltsverzeichnis === %
  	\tableofcontents
  	\newpage
  	
	% === Nomenklatur === %
	\pagenumbering{roman}\setcounter{page}{1}
  	\addcontentsline{toc}{chapter}{Nomenklaturverzeichnis}		% Größenverzeichnis ins Inhaltsverzeichnis
  	\begin{spacing}{1.2}
  		\printnomenclature
  	\end{spacing}  
  	\newpage
 
  	% === Inhalt === %
  	\pagenumbering{arabic}\setcounter{page}{1}
  	
  	\chapter{Einleitung}

So funktionieren Formeln in Latex:
\begin{center}
	\begin{equation}
		N = R^{*} \cdot f_{p} \cdot n_{e} \cdot f_{l} \cdot f_{i} \cdot f_{c} \cdot L 
	\end{equation}
\end{center}

% Einrücken von Text wird mit \noindent unterdrückt
\noindent So werden Bilder eingefügt:\\

\begin{figure}[h] % Das [h] sorgt dafür, dass das Bild exakt an dieser Stelle im Text eingefügt wird. Andernfalls sucht LaTeX automatisch nach einer Stelle in der Nähe wo das Bild am besten reinpasst
	\centering
	\label{figure:kepler}
	\includegraphics[width=0.2\textwidth]{kepler.jpg} % Breite = 0.2 x Seitenbreite
	\caption[Kepler (Titel im Abk.-Verzeichnis)]{Titel unter dem Bild - Quelle: \cite{key}}
	% als Quelle erscheint in der PDF ein [?], da der Eintrag "key" nicht in der literatur.bib Datei existiert
\end{figure}

\section{So wird eine neue Überschrift erstellt}
So werden Abbildungen/Tabellen/Abschnitte referenziert: Abb. \ref{figure:kepler}\\

\subsection{So wird eine neue Unterüberschrift erstellt}
Das sind die Absatzarten:\\
Einfacher Absatz\\\\
Absatz mit Abstand\\

Absatz mit Abstand, mit Einrücken\\\\

\noindent Bestimmte Zeichen wie z.B. \& müssen mit vorangehendem Backslash versehen werden, da sie ansonsten als LaTeX Steuerungszeichen interpretiert werden. ,,Gänsefüßchen`` gehen so...

\newpage

Alle verwendeten Symbole müssen ins das Nomenklaturverzeichnis eingetragen werden. Dazu werden folgende Befehle genutzt:

\nomenclature[AF]{$F$}{Kraft}{N}{}
\nomenclature[Am]{$m$}{Masse}{kg}{}




  	\chapter{Theoretische Grundlagen}
\label{chapter:theorie}
Um die aerodynamischen Größen, wie Auftrieb, Widerstand und deren Beiwerte ohne direkte Kraftmessung zu bestimmen, sind einige Formeln sowie theoretische Grundlagen erforderlich. Da für den Flugversuch nur der stationäre Sinkflug ohne Schub (Gleiten) relevant ist, wird im Folgendem nur dieser Zustand betrachtet. \\ \\
Zusätzlich werden folgende Vereinfachungen getroffen:
\begin{itemize}
	\item konst. Bahnwinkel: $\gamma = const.$ 
	\item Propellerschub gleicht Propellerwiderstand aus: $F=0$
	\item konst. Geschwindigkeit $V_{IAS} = const.$
	\item Instrumente sind auf Standardatmosphäre kalibriert
	\item Einbaufehler der Messinstrumente werden nicht berücksichtigt
	\item Vernachlässigung vom Wind
\end{itemize}

\vspace{1cm}
\section{Standardatmosphäre}
Für die Auswertung der Versuchsdaten ist es erforderlich Kenntnis über die Standardatmosphäre zu haben. Wichtige Werte sind die Temperatur, Dichte sowie der Druck in den jeweiligen Höhen. In Tab. \ref{tab:atm_0} sind diese Werte auf Meereshöhe angegeben.
\begin{table}[h]
	\centering
	\begin{tabular}{l}
		\hline
		$T_0=288,15$ $K$ \\
		$p_0=101325$ $Pa$ \\
		$\varrho_0 = 1,225$ $kg/m^3$ \\
		\hline		
	\end{tabular}
	\caption{Werte der Normatmosphäre auf Meereshöhe} \label{tab:atm_0}
\end{table}
Um diese Werte für verschiedene Höhen zu berechnen, können folgende Gleichungen benutzt werden. Diese gelten nur für den Bereich zwischen 0 und 11km Höhe.
\begin{equation}\label{eq:T_INA}
T=288,15K - 0,0065 \frac{K}{m} \cdot H
\end{equation}
\begin{equation}\label{eq:p_INA}
p=p_0\biggl[1-0,0065 \frac{K}{m} \cdot \frac{1}{288,15K} \cdot H\biggr]^{5,256}
\end{equation}
\begin{equation}\label{eq:roh_INA}
\varrho = \varrho_0 \biggl[1-0,0065 \frac{K}{m} \cdot \frac{1}{288,15K} \cdot H\biggr]^{4,256}
\end{equation}
\section{Stationärer Gleitflug}
Der Grundsatz des stationären Fluges ist, dass die am Flugzeug angreifenden Kräfte im Gleichgewicht stehen und somit das Flugzeug weder beschleunigt noch verzögert. Bei dem Gleitflug kommt hinzu, dass die Schubkraft gleich null ist.
\subsection{Widerstand und Auftrieb}
Das Kräftegleichgewicht mit den zuvor getroffenen Annahmen liefert:
\begin{equation}
\label{eq:kräftegleichgewichtWiderstand}
W + mg \cdot sin(\gamma) = 0
\end{equation}
\nomenclature[GG]{$\gamma$}{Bahnneigungswinkel}{[°]}{}%
\nomenclature[AM]{$m$}{Masse}{[$kg$]}{}%
\nomenclature[AG]{$g$}{Erdbeschleunigung}{[$m/s^2$]}{}%
\begin{equation}
\label{eq:kräftegleichgewichtAuftrieb}
A - mg \cdot cos(\gamma) = 0
\end{equation}
Abbildung \ref{figure:luftkraefte} verdeutlicht diese Zusammenhänge. \\  \\
Für spätere Berechnungen sind die Widerstands- sowie Auftriebsbeiwerte wesentlich wichtiger als die absoluten Werte.
\begin{equation}
C_W = \frac{-mg \cdot sin(\gamma)}{0,5  \cdot  \varrho  \cdot  S  \cdot  V^2}
\end{equation}
\begin{equation}
C_A = \frac{mg \cdot cos(\gamma)}{0,5  \cdot  \varrho  \cdot  S  \cdot  V^2}
\end{equation}
\subsection{Bahnwinkel und Gleitzahl}
Der nach oben hin positiv definierte Bahnwinkel $\gamma$ ergibt sich aus dem Kräftegleichgewicht zu:
\begin{equation}
tan(\gamma)=-\frac{W}{A}
\end{equation}
und unter Verwendung der Auftriebs- und Widerstandsbeiwerte zu:
\begin{equation}
tan(\gamma)=-\frac{C_W}{C_A}
\end{equation}
Durch trigonometrische Beziehungen ergibt sich wie in Abb. \ref{figure:luftkraefte} zu sehen:
\begin{equation}
\label{eq:bahnneigungswinkel}
sin(\gamma) = -\frac{w_g}{V}
\end{equation}
\nomenclature[AW]{$w_g$}{Sinkgeschwindigkeit}{[$m/s$]}{}%
\nomenclature[AV]{$V$}{Geschwindigkeit}{[$m/s$]}{}%
\nomenclature[AC]{$C_A$}{Auftriebsbeiwert}{[1]}{}%
\nomenclature[AC]{$C_W$}{Widerstandsbeiwert}{[1]}{}%
Um Aussagen über die aerodynamische Güte eines Flugzeuges treffen zu können wird die reziproke Gleitzahl $\varepsilon$ eingeführt. Sie ergibt sich aus dem Verhältnis von Widerstand und Auftrieb.
\begin{equation}
\varepsilon=\frac{W}{A}=-tan(\gamma)
\end{equation}
\nomenclature[GE]{$\varepsilon$}{reziproke Gleitzahl}{[1]}{}%

\section{Umrechnen der Versuchsdaten}
\subsection{Fluggeschwindigkeit}
Die bei dem Testflug abgelesene Fluggeschwindigkeit ($V_{IAS}$: indicated airspeed) muss vor der Auswertung in die reale Geschwindigkeit ($V_{TAS}$: true airspeed) umgerechnet werden.
\begin{equation} \label{eq:VTAS}
V_{TAS}=V_{IAS} \sqrt{\frac{\varrho_0}{\varrho_{real}}}
\end{equation} 
Die reale Dichte ergibt sich unter Annahme einer isobaren Zustandsänderung aus:
\begin{equation} \label{eq:roh_real}
\varrho_{real} = \varrho_{INA}  \cdot  \frac{T_{INA}}{T_{real}}
\end{equation}
Gl. \ref{eq:roh_real} in Gl. \ref{eq:VTAS} ergibt:
\begin{equation}
V_{TAS}=V_{IAS} \sqrt{\frac{\varrho_0}{\varrho_{INA}} \cdot \frac{T_{real}}{T_{INA}}}
\end{equation}
\nomenclature[AS]{$S$}{Flügelfläche}{[$m^2$]}{}%
\subsection{Höhendifferenz und Sinkgeschwindigkeit}
Da bei dem Testflug der Höhenmesser auf Standardatmosphäre eingestellt war, muss die angezeigte Höhe zuerst in die reale Höhe umgerechnet werden.
\begin{equation} \label{eq:deltaH_real}
\Delta H_{real} = \Delta H_{INA}  \cdot  \frac{\varrho_{INA}}{\varrho_{real}}
\end{equation}
Gl. \ref{eq:roh_real} in Gl. \ref{eq:deltaH_real} ergibt:
\begin{equation}
\Delta H_{real} = \Delta H_{INA}  \cdot  \frac{T_{real}}{T_{INA}}
\end{equation}
Um die Sinkgeschwindigkeit zu errechnen muss der Quotient aus zurückgelegter vertikaler Strecke mit der dazu benötigten Zeit gebildet werden:
\begin{equation}\label{eq:wg_real}
w_{g_{real}}=\frac{\Delta H_{real}}{\Delta t} = \frac{\Delta H_{INA}}{\Delta t}  \cdot  \frac{T_{real}}{T_{INA}}
\end{equation}


\newpage
  	\chapter{Versuchsdurchführung}
\label{chapter:versuch}

\newpage

Am Tag 21.05.2019 um 13:00 Uhr in Braunschweig Flughafen wurde der Versuch durchgeführt. Mit Pilot flogen 6 Personen zusammen. In Prinzipiell war die damalige Wetter nicht typisch geeignet für Flug, weil es  sanft windig und schwer bewölkt war. Die Temperatur war 288 $K$ und der Bodendruck war 1006 $hPa$. 

Der angewendete Flugzeug zum Versuch ist Do 128, der sich zu einem kleinen Flugzeug eignet. Die Spannbereite $b$ vom Do 128 ist 15,00 $m$ und die Spannfläche beträgt 29 ${m}^{2}$. Do 128 ist gar nicht so schwer bzw. die Rüstmasse ist nur 3188 $kg$. Dieser Flugzeug besitzt zwei Haupttanks, die jeweils 440 $lbs$ (199,58 $kg$) wiegen, und zwei Reservetanks, die jeweils 406 $lbs$ (184.16 $kg$) wiegen, und trägt maximal $V_{Kraftst,max}$ 1470 $l$ Kraftstoff, der mit Dichte $rho_{Kraftst}$ 0.784 $kg/l$ ist. 

Es werden vier stationäre Sinkflüge bei unterschiedlichen Fluggeschwindigkeiten durchgeführt. Da  es im stationäre Sinkflug keinen Schub gibt und Kraftgleichgewicht herrscht, braucht man $A$ und $W$ auszurechnen statt direkt zu messen.

In dem ersten Versuch stieg der Flugzeug in 5 Minuten in der Höhe 2000 $ft$ (609 $m$), sank mit Geschwindigkeit 80 $kts$ (41,16 $m/s$) in die Höhe 1450 $ft$ (441,96 $m$). Nach kurzer Zeit fing der Flugzeug dem Sinkflug von 2500 $ft$ (762 $m$) mit Geschwindigkeit 100 $kts$ (51,44 $m/s$) bis 2000 $ft$ (609,6 $m$). In den weiteren zwei Versuchen waren die Anfangshöhe und Endhöhe immer gleich wie bei zweitem Versuch bzw. 2500 $ft$ (762 $m$) und 2000 $ft$ (609,6 $m$) und der dritte Versuch war mit Geschwindigkeit 120 $kts$ (61,73 $m/s$) und der vierte Versuch war mit Geschwindigkeit 140 $kts$ (72,02 $m/s$). Die vier Versuchen dauerten jeweils 1 Minuten 38 Sekunden, 1 Minuten 8 Sekunden, 48 Sekunden und 31 Sekunden.


  	\include{Seiten/ergebnisse}
  	\chapter{Diskussion des Versuches}
\label{chapter:fazit}
\section{Diskussion von Marco}
\subsection{Höhenruder-Trimmkurve}

\subsection{Auftriebsbeiwert über den Anstellwinkel}
\subsection{LILIENTHAL-Polare}
Die Theorie besagt, dass $C_{W}^{*} = 2C_{W0}$ ist. Wendet man dies auf unsere Auswertung an Stellt man fest, dass dieser Zusammenhang dort nicht gilt. \\
Im Hinblick auf den Oswald-Faktor gibt es große Abweichungen zur Theorie. Für die DO 128-6 wurde ein Oswald-Faktor von $e=1,674$ berechnet. Dieser kann laut Definition allerdings maximal 1 sein. Da $e=f(k, \Lambda)$ gilt, und die Streckung eine rein geometrische Größe ist der von uns errechnete Widerstandsanstieg k nicht korrekt. 
\subsection{Widerstand über die Fluggeschwindigkeit}
Die Kurven der beiden Diagramme entsprechen qualitativ in etwa der Theorie. Im Vergleich zur DO 28 besitzt die DO 128-6 einen deutlich geringeren minimalen Widerstand und geringer optimale Geschwindigkeit, obwohl die DO128-6 deutlich schwerer ist als ihr Vorgänger. Um beide Flugzeuge allerdings wirklich miteinander Vergleichen zu können müsste man für gleiche Umgebungsbedingungen sorgen. Wettereinflüsse, sowie das Rechnen mit Vereinfachungen und Annahmen wie in Kap. \ref{chapter:theorie} beschrieben, können einen großen Einfluss auf die Ergebnisse haben.
\subsection{Staudruck und Fluggeschwindigkeit über dem Anstellwinkel}
\newpage

\section{Diskussion von Jens}
In diesem Labor haben wir gelernt, wie wir mit einer großen Menge an Daten arbeiten und diese möglichst effizient darstellen können. Gerade das Erstellen von Diagrammen hat einige Schwierigkeiten bereitet, welche dazu führten, dass wir deutlich mehr Zeit brauchten als ursprünglich eingeplant. So sind gewisse Fehler erst sehr spät aufgefallen, sodass diverse Diagramme immer wieder neu erstellt und darauf hin auch die Interpretationen leicht angepasst werden mussten. Dadurch haben wir jedoch intensiv den Umgang mit Plot-Programmen, aber vor allen Dingen auch das be- und umrechnen von aerodynamischen Größen gelernt. Gerade letzteres hat noch einmal das theoretische Wissen aus vergangen Semestern gestützt und erweitert.\\
Eine große Herausforderung war auch das Einschätzen der Validität von Daten. Viele Datensätze mussten von uns ignoriert werden, um ein möglichst konsistentes Ergebnis zu erhalten. Hierbei war es jedoch schwierig eine klare Grenze zu ziehen ab der man Werte als ,,Ausreißer`` ansieht und ignoriert.\\\\
Das Ziel war es letztlich aerodynamische Größen im Flugversuch zu ermitteln und auszuwerten, was uns zumindest in Ansätzen gelungen ist. Klar ist, dass die ermittelten Diagramme sicherlich nicht der Realität entsprechen. Dazu fehlte es an genaueren Daten und einer größeren Menge von Messdaten. Viel wichtiger ist jedoch, dass wir gelernt haben solche Kennwerte zu bestimmen und darzustellen, auch wenn uns möglicherweise Fehler und Ungenauigkeiten unterlaufen sind.

\subsubsubsection{Auftretende Fehler}
Bereits während des Versuches treten diverse Fehler auf, die auf ungenaue Messanzeigen oder auch das ungenaue Ablesen von Messanzeigen basieren. Als Beispiel sei hier die Temperaturmessung erwähnt: Allen voran besitzen Temperaturmesser eine hohe Zeitkonstante. Es dauert als schon recht lange, bis eine sich ändernde Temperatur auch tatsächlich auf dem Messgerät angezeigt wird. Bis nun auch noch der Protokollant die Temperatur von einer relativ ungenauen Skala abliest und niederschreibt vergeht wieder Zeit, was zu einem ungenauen Ergebnis beiträgt.\\
Auch im Verlauf des Auswertens der Daten treten jedoch Fehler auf. Bei der Do 28 war es zum Beispiel sehr unpräzise die Daten aus den gedruckten Messschrieben abzulesen. Ein Zugriff auf die Rohdaten hätte diesen Fehler minimieren können. Auch beim Berechnen und Umrechnen pflanzen sich Fehler unter anderem durch Runden von Zwischenergebnissen, aber auch durch vereinfachte Berechnungsformeln fort (als Beispiel sei hier die vereinfachende Annahme einer isobaren Zustandsänderung bei der Berechnung von Atmosphärendaten erwähnt).\\\\
All diese Fehler sorgen zum Schluss für ein ungenaues, vielleicht sogar unbrauchbares Ergebniss.

\newpage

\section{Diskussion von Philip}
Das Ziel dieses Labor war es die Durchführung eines Flugversuches mit anschließender Auswertung von Messdaten zu erlernen. Ein Versuch bedarf eine akribische Vorbereitung mit vorgefertigten Protokollen, damit man sich während des Versuchs auf das Wesentliche konzentrieren kann und so einen möglichst reibungslosen Verlauf ermöglichen kann. Man sollte im Vorfeld genaue Kenntnis darüber haben, wie man den Versuch durchführen will und was man dafür an Daten benötigt. \\\\
Um die Messdaten, egal ob selbst aufgezeichnet oder durch die Messschrieben entnommen, auswerten zu können, ist es nötig diese zuerst aufzubereiten. Dies geschieht im Wesentlichen durch die Berechnungen von aussagekräftigen Parametern mit anschließendem plotten in Tabellenkalkulationsprogrammen. Dafür brauchten wir die Kenntnisse, die wir in den letzten Semestern gewonnen hatten. Das Schwierige an der Auswertung der Daten waren einmal die großen Toleranzen, die durch das händische Ablesen der Daten im Flugzeug und dem Auswerten der Messchrieben entstanden sind. Zudem ist jede Rechnung aufgrund von Rundungen und Annahmen fehlerbehaftet. Es erfordert eine Wertung dieser Ergebnisse, welche sich für uns ohne Erfahrung recht schwierig erwiesen hat. Wichtig ist jedoch, dass das Arbeiten und der Umgang mit den Messdaten dazu geführt hat, unser Wissen zu festigen und zu erweitern.\\\\
Dadurch, dass die Versuche nicht unter Laborbedingungen durchgeführt werden können, erleben wir alle Umgebungseinflüsse, wie Wind und Wetter, hautnah. Dies bedeutet, dass man, um grobe aussagekräftige Daten zu erhalten, nur mit Mittelwerten und der Überführung in die Normatmosphäre arbeiten kann. Der Wind und die vielen weiteren Fehlerquellen zwingen einen dazu mit immer ungenaueren Ergebnissen zu Verfahren. \\\\
Die Auswertung der Daten brachte doch einige Probleme mit sich, wodurch eine angeregte Diskussion in der Gruppe entstand. Die Ergebnisse, die uns dann vorlagen entsprachen zum Teil nicht unseren Erwartungen, was in erster Linie Fragen aufwarf und ein Misstrauen gegenüber den Messdaten brachte. Dies hat sich auch nicht gelegt.
\\\\
Um die Fehler zu verkleinern, wäre es hilfreich, wenn die Messdaten der Do 28 zusätzlich zu den Diagrammen noch als Zahlenwerte geliefert würden. Sowie den Wetterdaten am Tag des Messfluges. Dadurch lässt sich zumindest grob ein Urteil darüber bilden, wie in etwa die Umgebungsbedingungen waren und was für ein Wind an diesem Tag geherrscht hat. 

\newpage

\section{Diskussion von Kilian}
\subsection{Höhenruder-Trimmkurve}

Im Allgemeinen erwartet man bei der Trimmung einen linearen Zusammenhang zwischen dem Winkel des Ausschlags des Höhenruders und dem induzierten Drehmoment. Folglich sollte auch der Zusammenhang von Trimmwinkel und Anstellwinkel linear sein. Für die Sinkflüge eins, zwei und vier gilt dies zumindest für die Stichproben innerhalb der Sinkflüge näherungsweise. Bei Sinkflug drei ist der dritte Wert leicht erhöht. Darüber hinaus unterscheidet sich die Wirkung der Trimmung auf den Anstellwinkel zwischen den Sinkflügel erheblich. Dies liegt möglicherweise an unterschiedlichen Wetterverhältnissen. Der im Versuch vernachlässigte Wind hat einen erheblichen Einfluss auf das nötige Drehmoment um einen bestimmten Anstellwinkel zu halten.\\
Ein Versuch im Windkanal mit skaliertem Modell könnte eine Wetterunabhängige Messreihe generieren.\\\\

\subsection{Auftriebsbeiwert über den Anstellwinkel}

In der Theorie folgen die Werte dieser Messreihe bei stabilen Flugzuständen einer Geraden. Wird ein bestimmter Anstellwinkel $\alpha({C_{A,max}})$ überschritten kommt es zum Strömungsabriss.\\
Die Daten aus den Stichproben geben diesen Sachverhalt im Bereich stabiler Flugzustände ohne nennenswerte Abweichungen wieder.\\\\

\subsection{LILIENTHAL-Polare}

Die qualitative Gestalt der LILIENTHAL-Polare sowohl für die Do-128 als auch die Do-28 entspricht der Theorie. Wie in der Interpretation erwähnt ist der Verlauf für die Do-128 etwas steiler als für die Do-28. Werte für die sich die beiden Flugzeuge im Wesentlichen unterschieden waren Masse und Flügelfläche. Trotz der deutlich höheren Masse und nur geringfügig höheren Flügelfläche erzielte die Do-128 bessere $C_{A}$ als die Do-28.\\
Auch hier können Unterschiedliche Wetterlagen zu veränderten Flugleistungen geführt haben und das Ergebnis maßgeblich verfälschen.
Interessant wäre eine Erörterung über den Einfluss der Flügelfläche, und ob der Unterschied von $1m^{2}$ oder die variable Wetterlage maßgeblich zu den unterschiedlichen Werten beigetragen hat. Auch hier kann ein skalierter Versuch im Windkanal eine Wetterunabhängige Versuchsreihe ermöglichen.\\\\

\subsection{Widerstand über die Fluggeschwindigkeit}

Der Werte der Do-128 entsprechen im groben einem zu erwartendem Verhältnis von $W$ und $V_{TAS}$. Auch hier verbirgt sich hinter dem Fehler das variable Wetter und ggf. Messfehler. Wie für alle wetterabhängigen Versuche ist ein Modellversuch im Windkanal ein geeignetes Mittel zur wetterunabhängigen Versuchsdurchführung.\\
Die in der Do-28 gemessenen Werte genügen als Ganzes genommen dem allgemeinen Zusammenhang nicht. Insbesondere die Stichprobenwerte für den dritten Sinkflug können mit den anderen Messwerten keine sinnvolle Basis für weitere Erkenntnisse sein.
Die Verknüpfung der Stichproben von Sinkflügen drei und vier oder Sinkflügen eins und vier könnte jedoch als Basis für weitere Betrachtungen sinnvoll sein, da diese zumindest die qualitative Gestalt des Zusammenhangs von $W$ und $V$ erfüllen.\\\\

\subsection{Staudruck und Fluggeschwindigkeit über dem Anstellwinkel}
Der Staudruck und die wahre Fluggeschwindigkeit sind formelhaft miteinander verknüpft (siehe Kapitel Theoretische Grundlagen). Konsistente Daten würden diesen Zusammenhang wiedergeben, indem Sie den Staudruck als betragsmäßig vielfaches der wahren Fluggeschwindigkeit zeigen. Dies ist für die Stichproben der Do-28 gegeben.
\newpage

\section{Diskussion von Wentao}
 Es gibt manche Messfehler und ist auch von äußeren Faktoren ab und zu beeinflußt, trotzdem ist der Versuch zusammenfassend gut geeignet um die Theorie und ähnelt sich auch den vorherigen Versuch vom $Do - 28$. Mit zahlreichen Daten von Messreihen verringert die Messfehler sich möglichst, aber im eigenen Versuch ist das Wertebereich nicht groß genügend für Messfehler zu verringern, obwohl die Plots vom Eigenversuch tatsächlich noch in Ordnung sind. Die Wetter, genaue Kenntnisse an der Aerodynamik des $Do -128$ und noch anderen äußere Faktoren, die wahrscheinlich die Messfehler verursacht, werden aber im Versuch vernachlässigt. Ein möglicher Lösung für die Messfehler zu vermeiden ist der Windkanal, um den Umgebungsbeeinfluss zu kontrollieren, oder mit einem Segelflugzeug, der relativ einfachere Aerodynamik besitzt.

Das Diagramm von Höhenruder - Trimmkurve zeigt näherungsweise eine lineare Beziehung zwischen $\alpha$ und $\eta$, welche auch erwartet wird. Es gibt viele kleine Abweichungen und eine Differenz von 1 Grad im zweiten Sinkflug, sodass man mit Mittelwert von vier Werten für Bestimmung ausrechnen muss, aber es beeinflusst das allgemeine Verlauf nicht.

Für den Verlauf des Auftriebsbeiwert über den Anstellwinkel ist eine Gerade zu erkennen, das auch das Normalverlauf entspricht. Bei einem bestimmten hoch Anstellwinkel soll es zum eine Strömungsabriss führen, das erst durch weiteren Messwerten bestimmt werden kann, in Stichproben werden aber nur die Daten für stationären Flug eingetragen.

Die zwei Lilienthal - Polare von Messreihen und eigener Flugversuch zeigen sich die ähnliche Verläufe und stimmen mit dem allgemeinen Verlauf überein. Das Verlauf vom $Do - 128$ ist relativ steiler wegen größeres Auftriebsbeiwert in niedrigerem Widerstandsbeiwert bzw. besserer Aufbau beim $Do - 128$ führt zu geringeren Widerstand. Alle Messdaten haben schwache Schwankungen und bleiben fast genau in der Kurve.

Beim Widerstand über der Fluggeschwindigkeit zeigen beide Bilder etwas Unterschiedliches. Die Plots vom $Do - 128$ stehen noch im maßgebenden Bereich. Beim dritten und vierten Sinkflüge vom $Do - 28$ sind große sinnlose Streuung aufzuweisen, obwohl es insgesamt eine sinnvolle Parabel bildet. Die optimale Geschwindigkeit $V_{opt}$ ist beim $Do - 28$ größer sowie der minimale Widerstand $W_{min}$.

Man erwartet eine parabelförmige Linie, die auch genau durch Messreihen gezeichnet werden, da die Wahre Geschwindigkeit und der Staudruck direkt durch spezifischen Messgerät aufgetragen, sind die Schwankungen viel kleiner. Der Staudruck hat einem biquadratischen Zusammenhang zwischen die Geschwindigkeit.
\newpage

   	% === Literaturverzeichnis === %
  	\addcontentsline{toc}{chapter}{Literaturverzeichnis}
    \bibliographystyle{unsrt}
    \bibliography{literatur}

	% === Abbildungsverzeichnis === %
	\addcontentsline{toc}{chapter}{Abbildungsverzeichnis}
	\listoffigures
	\newpage
  
	% === Tabellenverzeichnis === %
	\addcontentsline{toc}{chapter}{Tabellenverzeichnis}
	\listoftables						
	\newpage
  
	% === Anhang === %
	\begin{appendix}
	 	% SACHEN FÜR DEN ANHANG HIER INCLUDIEREN
	%\include{Seiten/anhang1}
	\end{appendix}

\end{document}
