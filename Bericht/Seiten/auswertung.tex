\chapter{Auswertung der Messdaten}
\label{chapter:auswertung}

\section{Do 128-6 (eigener Flugversuch)}
Wie die Treibstoffmasse müssen auch die anderen abgelesene Daten ($T_{real}$ in °$C$, $H_{INA}$ in $ft$, $V_{IAS}$ in $kn$) vor der Analyse in SI Basiseinheiten (°$C  \rightarrow K$, $ft \rightarrow m$, $kn \rightarrow m/s$) umgerechnet werden. In Tab. \ref{tab:umrechnung_einheiten} sind diese Umrechnungen aufgeführt.\\

\begin{table}[h]
	\centering
	\begin{tabular}{l l}
		\hline
		Höhe 			& $1$ $ft = 0,3048\,m$ 								\\
		Geschwindigkeit	&   $1\,kn = 0,51\,m/s$								\\
		Temperatur 		&  $t\,^\circ C = t\frac{K}{^\circ C}+273,15\,K$	\\
		\hline		
	\end{tabular}
	\caption{Umrechnungstabelle auf SI Basiseinheiten} \label{tab:umrechnung_einheiten}
\end{table}

\noindent Die in Tabelle \ref{tab:messergebnisse} angegebenen Messergebnisse lauten mit diesen Umrechnungen wie folgt (Hinweis: Aufgrund der ohnehin recht ungenauen Temperaturmessung, wurde statt der 273,15 K der Einfachheit halber mit 273 K umgerechnet).

\begin{table}[h]
	\centering
	\begin{tabular}{|c|c|c|c|c|}
		\hline
		& \multicolumn{1}{l|}{\textbf{$V_{IAS}$}} & \multicolumn{1}{l|}{\textbf{$\Delta t$}} & \multicolumn{1}{l|}{\textbf{$T_{real,start}$}} & \multicolumn{1}{l|}{\textbf{$T_{real,ende}$}} \\ \hline
		1 & 40,8 m/s         & 98 s         & 285,0 K         & 287,5 K    \\ \hline
		2 & 51,0 m/s         & 68 s         & 285,0 K         & 288,0 K    \\ \hline
		3 & 61,2 m/s         & 48 s         & 286,0 K         & 289,0 K    \\ \hline
		4 & 71,4 m/s         & 31 s         & 286,0 K         & 288,0 K    \\ \hline
	\end{tabular}
	\caption{Do 128-6 Messergebnisse in SI-Einheiten} \label{tab:messergebnisse-si}
\end{table}

\noindent Da, bedingt durch das Wetter, am Versuchstag keine Standardatmosphärenbedingungen herrschten, muss die in Kapitel 2 hergeleitete Formel \ref{eq:wg_real} verwendet werden, um die wahre Sinkgeschwindigkeit $w_{g,real}$ für die einzelnen Sinkflüge bestimmen zu können. Dazu ist eine berechnete theoretische Temperatur aus der Normatmosphäre nötig, nämlich die Temperatur, welche laut Normatmosphäre bei Standardatmosphärenbedingungen in der jeweiligen Höhe herrschen würde $T_{INA}$. Weiterhin muss auch das Höhenintervall zwischen 2500 ft und 1500 ft in Metern umgerechnet werden und man erhält ein $\Delta H_{INA}$ von 304,8 m. Beispielhaft sei im folgenden die Leistungsdaten Berechnung für Sinkflug 1 demonstriert:\\

\noindent 
\underline{Wahre Sinkgeschwindigkeit bestimmen}

\begin{equation}
T_{INA}(2500m) = 288,15\,K - 0,0065 \frac{K}{m} \cdot 2500\,m = 271,9\,K
\end{equation}
\begin{equation}
T_{INA}(1500m) = 288,15\,K - 0,0065 \frac{K}{m} \cdot 1500\,m = 278,4\,K
\end{equation}
\begin{equation}
T_{INA,mittel} = \frac{271,9\,K + 278,4\,K}{2} = 275,15\,K
\end{equation}
\begin{equation}
T_{real,m} = \frac{285,0 K + 287,5 K}{2} =	286,25\,K
\end{equation}
\begin{equation}
w_{g,real} = \frac{-304,8\,m}{98\,s} \cdot \frac{286,25\,K}{275,15\,K} = - 3,24\,m/s
\end{equation}

% TODO Formeln aus THEORIE Teil referenzieren
\vspace{5mm}
\noindent Neben der realen Sinkgeschwindigkeit muss auch die angzeigte Geschwindigkeit $V_{IAS}$ in die wahre Geschwindigkeit $V_{TAS}$ umgerechnet werden.\\


\noindent
\underline{Wahre Geschwindigkeit bestimmen}

\begin{equation}
\rho_{INA}(2500ft) = \rho_0 * (1-0,065\cdot \frac{H}{T_0})^{4,256} = 1,225 kg/m^3 * (1-0,0065\cdot \frac{762 m}{288,15 K})^{4,256}
\end{equation}
\begin{equation}
\rho_{INA}(2500ft) = 1,1378 kg/m^3
\end{equation}
\begin{equation}
\rho_{INA}(1500ft) = 1,1721 kg/m^3
\end{equation}
\begin{equation}
\rho_{INA,mittel} = \frac{1,1721 kg/m^3+1,1378 kg/m^3}{2} = 1,15495 kg/m^3
\end{equation}

\vspace{5mm}

\begin{equation}
\rho_{real} = \rho_{INA,mittel} * \frac{T_{INA}}{T_{real}} = 1,11 kg/m^3
\end{equation}

\vspace{5mm}

\begin{equation}
V_{TAS} = V_{IAS}\sqrt{\frac{\rho_0}{\rho_{real}}} =42,86 m/s
\end{equation}


\vspace{5mm} \noindent
\underline{Luftkräfte und Beiwerte bestimmen}

\noindent Nun wird aus der wahren Sinkgeschwindigkeit noch der Längsneigungswinkel $\gamma$ mit $\gamma = asin(-w_g/V_{TAS})$ zu 4,33° bestimmt, sodass letztlich die Luftkräfte ausgerechnet werden können:

\begin{equation}
W = - m\cdot g \cdot sin(\gamma) = -3216.76 N
\end{equation}
\begin{equation}
A = m\cdot g \cdot cos(\gamma) = 42486.17 N
\end{equation}

\vspace{5mm} \noindent
Und aus den Kräften zum Schluss die aerodynamischen Beiwerte unter Nutzung der Flügelfläche S aus Tabelle \ref{tab:do128}.

\begin{equation}
C_W = \frac{2*W*T_real}{\rho_{INA} * T_{INA} * V_{TAS}^2 * S} = 0,10879
\end{equation}
\begin{equation}
C_A = \frac{2*A*T_real}{\rho_{INA} * T_{INA} * V_{TAS}^2 * S} = 1,43689
\end{equation}

\vspace{5mm} \noindent
Die Ergebnisse für alle Sinkflüge sind in folgenden Tabellen einsehbar.

\begin{table}[h]
	\centering
	\begin{tabular}{|c|c|c|c|c|c|}
		\hline
		& \multicolumn{1}{l|}{\textbf{$T_{real,mittel} {[}K{]}$}} & \multicolumn{1}{l|}{\textbf{$T_{INA,mittel} {[}K{]}$}} & \textbf{$wg_{real} {[}m/s{]}$} & \textbf{$V_{TAS} {[}m/s{]}$} & $\gamma$ $[^\circ]$ \\ \hline
		1 & 286,25                                        & 275,15                                       & - 3,236                     & 42,858                  & 4,33                   \\ \hline
		2 & 286,5                                         & 275,15                                       & - 4,667                    & 53,596                  & 5,00                   \\ \hline
		3 & 287,5                                         & 275,15                                       & - 6,635                    & 64,428                  & 5,91                   \\ \hline
		4 & 287                                           & 275,15                                       & - 10,256                   & 75,100                  & 7,85                   \\ \hline
	\end{tabular}
	%\caption{Auswertung des Do 128-6 Versuches}
\end{table}

\begin{table}[h]
	\centering
	\begin{tabular}{|l|l|l|l|l|}
		\hline
		& \textbf{$W {[}N{]}$} & \textbf{$A {[}N{]}$} & \textbf{$C_w$} & \textbf{$C_a$} \\ \hline
		1 & -3216,76           & 42486,17           & 0,10879     & 1,4368      \\ \hline
		2 & -3701,26           & 42341,83           & 0,08011     & 0,9164      \\ \hline
		3 & -4367,02           & 42179,24           & 0,06564     & 0,6340      \\ \hline
		4 & -5780,56           & 41933,10           & 0,06383     & 0,4630      \\ \hline
	\end{tabular}
	\caption{Auswertung des Do 128-6 Versuches}
\end{table}








\vspace{1cm}
\section{Do 28 (Messchriebe)}

\newpage