\chapter{Einleitung}

So funktionieren Formeln in Latex:
\begin{center}
	\begin{equation}
		N = R^{*} \cdot f_{p} \cdot n_{e} \cdot f_{l} \cdot f_{i} \cdot f_{c} \cdot L 
	\end{equation}
\end{center}

% Einrücken von Text wird mit \noindent unterdrückt
\noindent So werden Bilder eingefügt:\\

\begin{figure}[h] % Das [h] sorgt dafür, dass das Bild exakt an dieser Stelle im Text eingefügt wird. Andernfalls sucht LaTeX automatisch nach einer Stelle in der Nähe wo das Bild am besten reinpasst
	\centering
	\label{figure:kepler}
	\includegraphics[width=0.2\textwidth]{kepler.jpg} % Breite = 0.2 x Seitenbreite
	\caption[Kepler (Titel im Abk.-Verzeichnis)]{Titel unter dem Bild - Quelle: \cite{key}}
	% als Quelle erscheint in der PDF ein [?], da der Eintrag "key" nicht in der literatur.bib Datei existiert
\end{figure}

\section{So wird eine neue Überschrift erstellt}
So werden Abbildungen/Tabellen/Abschnitte referenziert: Abb. \ref{figure:kepler}\\

\subsection{So wird eine neue Unterüberschrift erstellt}
Das sind die Absatzarten:\\
Einfacher Absatz\\\\
Absatz mit Abstand\\

Absatz mit Abstand, mit Einrücken\\\\

\noindent Bestimmte Zeichen wie z.B. \& müssen mit vorangehendem Backslash versehen werden, da sie ansonsten als LaTeX Steuerungszeichen interpretiert werden. ,,Gänsefüßchen`` gehen so...

\newpage

Alle verwendeten Symbole müssen ins das Nomenklaturverzeichnis eingetragen werden. Dazu werden folgende Befehle genutzt:

\nomenclature[AF]{$F$}{Kraft}{N}{}
\nomenclature[Am]{$m$}{Masse}{kg}{}



