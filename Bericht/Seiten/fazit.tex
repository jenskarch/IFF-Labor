\chapter{Fazit und Fehlerdiskussion}
\label{chapter:fazit}
\section{Diskussion von Marco}
\newpage

\section{Diskussion von Jens}
\newpage

\section{Diskussion von Philip}
\newpage

\section{Diskussion von Kilian}
\subsection{Höhenruder-Trimmkurve}

Im Allgemeinen erwartet man bei der Trimmung einen linearen Zusammenhang zwischen dem Winkel des Ausschlags des Höhenruders und dem induzierten Drehmoment. Folglich sollte auch der Zusammenhang von Trimmwinkel und Anstellwinkel linear sein. Für die Sinkflüge eins, zwei und vier gilt dies zumindest für die Stichproben innerhalb der Sinkflüge näherungsweise. Bei Sinkflug drei ist der dritte Wert leicht erhöht. Darüber hinaus unterscheidet sich die Wirkung der Trimmung auf den Anstellwinkel zwischen den Sinkflügel erheblich. Dies liegt möglicherweise an unterschiedlichen Wettererhältnissen. Der im Versuch vernachlässigte Wind hat einen erheblichen Einfluss auf das nötige Drehmoment um einen bestimmten Anstellwinkel zu halten.\\
Ein Versuch im Windkanal mit skaliertem Modell könnte eine Wetterunabhängige Messreihe generieren.\\\\

\subsection{Auftriebsbeiwert über den Anstellwinkel}

In der Theorie folgen die Werte dieser Messreihe bei stabilen Flugzuständen einer Geraden. Wird ein bestimmter Anstellwinkel $\alpha({C_{A,max}})$ überschritten kommt es zum Strömungsabriss.\\
Die Daten aus den Stichproben geben diesen Sachverhalt im Bereich stabiler Flugzustände ohne nennenswerte Abweichungen wieder.\\\\

\subsection{LILIENTHAL-Polare}

Die qualitative Gestalt der LILIENTHAL-Polare sowohl für die Do-128 als auch die Do-28 entspricht der Theorie. Wie in der Interpretation erwähnt ist der Verlauf für die Do-128 etwas steiler als für die Do-28. Werte für die sich die beiden Flugzeuge im Wesentlichen unterschieden waren Masse und Flügelfläche. Trotz der deutlich höheren Masse und nur geringfügig höheren Flügelfläche erzielte die Do-128 bessere $C_{A}$ als die Do-28.\\
Auch hier können Unterschiedliche Wetterlagen zu veränderten Flugleistungen geführt haben und das Ergebnis maßgeblich verfälschen.
Interessant wäre eine Erörterung über den Einfluss der Flügelfläche, und ob der Unterschied von $1m^{2}$ oder die variable Wetterlage maßgeblich zu den unterschiedlichen Werten beigetragen hat. Auch hier kann ein skalierter Versuch im Windkanal eine Wetterunabhängige Versuchsreihe ermöglichen.\\\\

\subsection{Widerstand über die Fluggeschwindigkeit}

Der Werte der Do-128 entsprechen im groben einem zu erwartendem Verhältnis von $W$ und $V_{TAS}$. Auch hier verbirgt sich hinter dem Fehler das variable Wetter und ggf. Messfehler. Wie für alle wetterabhängigen Versuche ist ein Modellversuch im Windkanal ein geeignetes Mittel zur wetterunabhängigen Versuchsdurchführung.\\
Die in der Do-28 gemessenen Werte genügen als Ganzes genommen dem allgemeinen Zusammenhang nicht. Insbesondere die Stichprobenwerte für den dritten Sinkflug können mit den anderen Messwerten keine sinnvolle Basis für weitere Erkenntnisse sein.
Die Verknüpfung der Stichproben von Sinkflügen drei und vier oder Sinkflügen eins und vier könnte jedoch als Basis für weitere Betrachtungen sinnvoll sein, da diese zumindest die qualitative Gestalt des Zusammenhangs von $W$ und $V$ erfüllen.\\\\

\subsection{Staudruck und Fluggeschwindigkeit über dem Anstellwinkel}
Der Staudruck und die wahre Fluggeschwindigkeit sind formelhaft miteinander verknüpft (siehe Kapitel Theoretische Grundlagen). Konsistente Daten würden diesen Zusammenhang wiedergeben, indem Sie den Staudruck als betragsmäßig vielfaches der wahren Fluggeschwindigkeit zeigen. Dies ist für die Stichproben der Do-28 gegeben.
\newpage

\section{Diskussion von Wentao}
\newpage