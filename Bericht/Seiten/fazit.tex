\chapter{Diskussion des Versuches}
\label{chapter:fazit}
\section{Diskussion von Marco}
\subsection{Höhenruder-Trimmkurve}

\subsection{Auftriebsbeiwert über den Anstellwinkel}
\subsection{LILIENTHAL-Polare}
Die Theorie besagt, dass $C_{W}^{*} = 2C_{W0}$ ist. Wendet man dies auf unsere Auswertung an Stellt man fest, dass dieser Zusammenhang dort nicht gilt. \\
Im Hinblick auf den Oswald-Faktor gibt es große Abweichungen zur Theorie. Für die DO 128-6 wurde ein Oswald-Faktor von $e=1,674$ berechnet. Dieser kann laut Definition allerdings maximal 1 sein. Da $e=f(k, \Lambda)$ gilt, und die Streckung eine rein geometrische Größe ist der von uns errechnete Widerstandsanstieg k nicht korrekt. 
\subsection{Widerstand über die Fluggeschwindigkeit}
Die Kurven der beiden Diagramme entsprechen qualitativ in etwa der Theorie. Im Vergleich zur DO 28 besitzt die DO 128-6 einen deutlich geringeren minimalen Widerstand und geringer optimale Geschwindigkeit, obwohl die DO128-6 deutlich schwerer ist als ihr Vorgänger. Um beide Flugzeuge allerdings wirklich miteinander Vergleichen zu können müsste man für gleiche Umgebungsbedingungen sorgen. Wettereinflüsse, sowie das Rechnen mit Vereinfachungen und Annahmen wie in Kap. \ref{chapter:theorie} beschrieben, können einen großen Einfluss auf die Ergebnisse haben.
\subsection{Staudruck und Fluggeschwindigkeit über dem Anstellwinkel}
\newpage

\section{Diskussion von Jens}
In diesem Labor haben wir gelernt, wie wir mit einer großen Menge an Daten arbeiten und diese möglichst effizient darstellen können. Gerade das Erstellen von Diagrammen hat einige Schwierigkeiten bereitet, welche dazu führten, dass wir deutlich mehr Zeit brauchten als ursprünglich eingeplant. So sind gewisse Fehler erst sehr spät aufgefallen, sodass diverse Diagramme immer wieder neu erstellt und darauf hin auch die Interpretationen leicht angepasst werden mussten. Dadurch haben wir jedoch intensiv den Umgang mit Plot-Programmen, aber vor allen Dingen auch das be- und umrechnen von aerodynamischen Größen gelernt. Gerade letzteres hat noch einmal das theoretische Wissen aus vergangen Semestern gestützt und erweitert.\\
Eine große Herausforderung war auch das Einschätzen der Validität von Daten. Viele Datensätze mussten von uns ignoriert werden, um ein möglichst konsistentes Ergebnis zu erhalten. Hierbei war es jedoch schwierig eine klare Grenze zu ziehen ab der man Werte als ,,Ausreißer`` ansieht und ignoriert.\\\\
Das Ziel war es letztlich aerodynamische Größen im Flugversuch zu ermitteln und auszuwerten, was uns zumindest in Ansätzen gelungen ist. Klar ist, dass die ermittelten Diagramme sicherlich nicht der Realität entsprechen. Dazu fehlte es an genaueren Daten und einer größeren Menge von Messdaten. Viel wichtiger ist jedoch, dass wir gelernt haben solche Kennwerte zu bestimmen und darzustellen, auch wenn uns möglicherweise Fehler und Ungenauigkeiten unterlaufen sind.

\subsubsubsection{Auftretende Fehler}
Bereits während des Versuches treten diverse Fehler auf, die auf ungenaue Messanzeigen oder auch das ungenaue Ablesen von Messanzeigen basieren. Als Beispiel sei hier die Temperaturmessung erwähnt: Allen voran besitzen Temperaturmesser eine hohe Zeitkonstante. Es dauert als schon recht lange, bis eine sich ändernde Temperatur auch tatsächlich auf dem Messgerät angezeigt wird. Bis nun auch noch der Protokollant die Temperatur von einer relativ ungenauen Skala abliest und niederschreibt vergeht wieder Zeit, was zu einem ungenauen Ergebnis beiträgt.\\
Auch im Verlauf des Auswertens der Daten treten jedoch Fehler auf. Bei der Do 28 war es zum Beispiel sehr unpräzise die Daten aus den gedruckten Messschrieben abzulesen. Ein Zugriff auf die Rohdaten hätte diesen Fehler minimieren können. Auch beim Berechnen und Umrechnen pflanzen sich Fehler unter anderem durch Runden von Zwischenergebnissen, aber auch durch vereinfachte Berechnungsformeln fort (als Beispiel sei hier die vereinfachende Annahme einer isobaren Zustandsänderung bei der Berechnung von Atmosphärendaten erwähnt).\\\\
All diese Fehler sorgen zum Schluss für ein ungenaues, vielleicht sogar unbrauchbares Ergebniss.

\newpage

\section{Diskussion von Philip}
\newpage

\section{Diskussion von Kilian}
\subsection{Höhenruder-Trimmkurve}

Im Allgemeinen erwartet man bei der Trimmung einen linearen Zusammenhang zwischen dem Winkel des Ausschlags des Höhenruders und dem induzierten Drehmoment. Folglich sollte auch der Zusammenhang von Trimmwinkel und Anstellwinkel linear sein. Für die Sinkflüge eins, zwei und vier gilt dies zumindest für die Stichproben innerhalb der Sinkflüge näherungsweise. Bei Sinkflug drei ist der dritte Wert leicht erhöht. Darüber hinaus unterscheidet sich die Wirkung der Trimmung auf den Anstellwinkel zwischen den Sinkflügel erheblich. Dies liegt möglicherweise an unterschiedlichen Wetterverhältnissen. Der im Versuch vernachlässigte Wind hat einen erheblichen Einfluss auf das nötige Drehmoment um einen bestimmten Anstellwinkel zu halten.\\
Ein Versuch im Windkanal mit skaliertem Modell könnte eine Wetterunabhängige Messreihe generieren.\\\\

\subsection{Auftriebsbeiwert über den Anstellwinkel}

In der Theorie folgen die Werte dieser Messreihe bei stabilen Flugzuständen einer Geraden. Wird ein bestimmter Anstellwinkel $\alpha({C_{A,max}})$ überschritten kommt es zum Strömungsabriss.\\
Die Daten aus den Stichproben geben diesen Sachverhalt im Bereich stabiler Flugzustände ohne nennenswerte Abweichungen wieder.\\\\

\subsection{LILIENTHAL-Polare}

Die qualitative Gestalt der LILIENTHAL-Polare sowohl für die Do-128 als auch die Do-28 entspricht der Theorie. Wie in der Interpretation erwähnt ist der Verlauf für die Do-128 etwas steiler als für die Do-28. Werte für die sich die beiden Flugzeuge im Wesentlichen unterschieden waren Masse und Flügelfläche. Trotz der deutlich höheren Masse und nur geringfügig höheren Flügelfläche erzielte die Do-128 bessere $C_{A}$ als die Do-28.\\
Auch hier können Unterschiedliche Wetterlagen zu veränderten Flugleistungen geführt haben und das Ergebnis maßgeblich verfälschen.
Interessant wäre eine Erörterung über den Einfluss der Flügelfläche, und ob der Unterschied von $1m^{2}$ oder die variable Wetterlage maßgeblich zu den unterschiedlichen Werten beigetragen hat. Auch hier kann ein skalierter Versuch im Windkanal eine Wetterunabhängige Versuchsreihe ermöglichen.\\\\

\subsection{Widerstand über die Fluggeschwindigkeit}

Der Werte der Do-128 entsprechen im groben einem zu erwartendem Verhältnis von $W$ und $V_{TAS}$. Auch hier verbirgt sich hinter dem Fehler das variable Wetter und ggf. Messfehler. Wie für alle wetterabhängigen Versuche ist ein Modellversuch im Windkanal ein geeignetes Mittel zur wetterunabhängigen Versuchsdurchführung.\\
Die in der Do-28 gemessenen Werte genügen als Ganzes genommen dem allgemeinen Zusammenhang nicht. Insbesondere die Stichprobenwerte für den dritten Sinkflug können mit den anderen Messwerten keine sinnvolle Basis für weitere Erkenntnisse sein.
Die Verknüpfung der Stichproben von Sinkflügen drei und vier oder Sinkflügen eins und vier könnte jedoch als Basis für weitere Betrachtungen sinnvoll sein, da diese zumindest die qualitative Gestalt des Zusammenhangs von $W$ und $V$ erfüllen.\\\\

\subsection{Staudruck und Fluggeschwindigkeit über dem Anstellwinkel}
Der Staudruck und die wahre Fluggeschwindigkeit sind formelhaft miteinander verknüpft (siehe Kapitel Theoretische Grundlagen). Konsistente Daten würden diesen Zusammenhang wiedergeben, indem Sie den Staudruck als betragsmäßig vielfaches der wahren Fluggeschwindigkeit zeigen. Dies ist für die Stichproben der Do-28 gegeben.
\newpage

\section{Diskussion von Wentao}
\newpage