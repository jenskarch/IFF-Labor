\chapter{Interpretation der Ergebnisse}
\label{chapter:interpretation}

\section{Analyse durch Marco De Gaetano}
\subsection{Höhenruder-Trimmkurve}
\subsection{Auftriebsbeiwert über den Anstellwinkel}
\subsection{LILIENTHAL-Polare}
\subsection{Widerstand über die Fluggeschwindigkeit}
\subsection{Staudruck und Fluggeschwindigkeit über dem Anstellwinkel}
\newpage

\section{Analyse durch Jens Karch}
\subsection{Höhenruder-Trimmkurve}
Bei der Höhenruder-Trimmkurve in Abbildung \ref{abb:do28_alpha_eta} handelt es sich um den Anstellwinkel $\alpha$, welcher über den Höhenruderausschlag $\eta$ aufgetragen wird. Wie bei allen folgenden Plots, wurden auch von diesem nur eine Auswahl an den aus Kapitel \ref{chapter:auswertung} bekannten Ergebnissen für die graphische Auswertung heran gezogen, da einige Messpunkte unrealistische Ergebnisse liefern, welche die Regressionsqualität verschlechtern würden.\\
Offensichtlich ergibt sich für die Regressionslinie ein linearer Verlauf, welche mit folgender Formel angenähert werden kann.\\

\begin{equation}
\alpha(\eta) = -2,44\eta + 2,76
\end{equation} 

\vspace{5mm} \noindent
Direkt ablesbar daraus ist der Y-Achsen Schnittpunkt bei $y = 2,76$. Dieser Wert entspricht dem Anstellwinkel in Grad, der sich einstellt, wenn das Höhenruder sich in Neutralstellung befindet. Der Nullpunkt kann durch einfache Mathematik zu $x=1.13$ bestimmt werden, und beschreibt den Höhenruderausschlag in Grad, der einzustellen ist um einen Anstellwinkel von 0 $^\circ$ zu erhalten.\\
Generell handelt es sich bei der Kurve um eine Gerade mit negativer Steigung, was bedeutet, dass ein positiver Höhenruderausschlag eine Verringerung des Anstellwinkels zur Folge hat. Diese Ergebnisse decken sich gut mit der Theorie.\\\\

\subsection{Auftriebsbeiwert über den Anstellwinkel}
Der Auftriebsbeiwert $C_A$ über den Anstellwinkel $\alpha$ (Abbildung \ref{abb:do28_ca_alpha}) wird auch ,,aufgelöste Polare`` genannt. An diesem Diagramm lässt sich ablesen wie sich der Auftrieb über den Anstellwinkel verändert, aber vor allem, ab welchem Anstellwinkel ein Strömungsabriss zu erwarten ist. Je höher der Anstellwinkel ist, desto näher kommt man dem Bereich des Strömungsabrisses. Die Kenntnis der Eigenschaften eines Flugzeugs hinsichtlich dieses kritischen Bereichs ist von äußerster Relevanz.\\
In unserem Versuch kann diese Kurve, welche bei hohen Anstellwinkeln gewöhnlich nach unten abknickt (Strömungsabriss), allerdings nur in ihrem linearen Bereich angenähert werden, da uns nicht genügend Daten im hohen Anstellwinkelbereich zur Verfügung stehen um diesen ausreichend abbilden zu können. Dementsprechend ergibt sich der lineare Teil der aufgelösten Polare zu folgender Formel.\\

\begin{equation}  \label{eq:versuch_ca}
C_A(\alpha) = 0,11\alpha + 0,2
\end{equation}

\vspace{3mm} \noindent
Wieder kann hier der Y-Achsen Schnittpunkt bestimmt werden zu $y=0,2$. Bei diesem Wert handelt es sich um den Auftriebsbeiwert bei keinem Anstellwinkel $C_{A0}$. \\
Allgemein folgt die Gleichung $C_A(\alpha)$ der folgenden Form.\\

\begin{equation}  \label{eq:theorie_ca}
C_A(\alpha) = C_{A\alpha}(\alpha - \alpha_0)=C_{A\alpha}\alpha - \alpha_0 C_{A\alpha}
\end{equation}

\vspace{3mm} \noindent
Die übrigen zwei Parameter können so durch die Information dass es sich beim Nullauftriebsanstellwinkel $\alpha_0$ um den Nullpunkt von Gleichung \ref{eq:versuch_ca} handelt, sowie durch Koeffizientenvergleich dieser Gleichung mit der aus der Theorie bekannten Gleichung \ref{eq:theorie_ca}, ermittelt werden. Es ergibt sich:\\

\begin{itemize}
	\item $\alpha_0=-1.82 ^\circ$
	\item $C_{A\alpha}= 0,11$
\end{itemize}

\vspace{5mm} \noindent

\subsection{LILIENTHAL-Polare}
Es wurden insgesamt zwei LILIENTHAL-Polare erstellt. Einmal für das Versuchsflugzeug Do 128-6 in Abb. \ref{abb:do128_lilienthal} und einmal für die Do 28 in Abb. \ref{abb:do28_lilienthal}.\\
\subsubsection{Do 128-6}
Für die Do 128-6 standen vier Messpunkte zur Verfügung um einen quadratischen Verlauf zu approximieren.\\

\begin{equation}
C_W=-164C_{A}^2+48,7C_A-1,9
\end{equation}

\vspace{3mm} \noindent
Aus dieser Polarengleichung lassen sich diverse charakteristische Leistungsgrößen eines Flugzeugs ablesen. Die Parameter $k=-164$, $b=48,7$ \& $C_{W0}=-1,9$ können direkt als Koeffizienten abgelesen werden. Weiterhin kann der Betriebspunkt des besten Gleitens bei $C_{A}^* \approx 1,4$ \& $C_{W}^* \approx 0,108$ grafisch bestimmt werden. Dazu gehört eine minimale reziproke Gleitzahl:\\

\begin{equation}
\epsilon_{min}=\frac{C_W^*}{C_A^*}=0,0771
\end{equation}

\vspace{3mm} \noindent
Der Kehrwert dieser Zahl entspricht der besten Gleitzahl dieses Flugzeugs, welche also bei etwa \textbf{13} liegt. Aus einem Kilometer Höhe könnte das Flugzeug also theoretisch 13 km weit gleiten. Das entspricht einem Gleitwinkel von $\gamma =arctan(-C_W^*/C_A^*)=-4,4^\circ$. Moderne Segelflugzeuge besitzen eine Gleitzahl von 40 und mehr, allerdings besitzen deren Profile eine deutlich höhere aerodynamische Güte. Für ein Motorflugzeug scheint diese Gleitzahl schon durchaus realistisch zu sein, eher sogar noch zu hoch (zu gut).\\\\
Weiterhin lässt sich nun der Oswald-Faktor e bestimmen, für den jedoch vorher die Streckung $\Lambda$ des Profils nötig ist. Diese berechnet sich als Quotient von der Flügelspannweite b zum Quadrat durch die Flügelbezugsfläche S. Mit den Werten aus Tab. \ref{tab:do128} ergibt sich dann $\Lambda=8,338$. Jetzt lässt sich der Oswald-Faktor durch folgende Gleichung bestimmen.\\

\begin{equation}  
e=\frac{1}{k\pi \Lambda}=
\end{equation}

\vspace{3mm} \noindent


\subsection{Widerstand über die Fluggeschwindigkeit}
\subsection{Staudruck und Fluggeschwindigkeit über dem Anstellwinkel}
\newpage

\section{Analyse durch Philip Margenfeld}
\subsection{Höhenruder-Trimmkurve}
Die Höhenruder-Trimmkurve zeigt den Anstellwinkel $\alpha$ der Do 28 über den Klappenausschlag $\eta$ des Höhenruders. Dargestellt ist dies in Abbildung \ref{abb:do28_alpha_eta}. Ein negativer Klappenausschlag sorgt für ein Nickmoment um den Schwerpunkt der Do 28 führt damit zu einer Änderung des Anstellwinkels $\alpha$. Abbildung \ref{abb:do28_alpha_eta} zeigt einen linear negativen Verlauf, der sich mit der Gleichung
\begin{equation}
\alpha(\eta) = -2,44\eta + 2,76
\end{equation}
darstellen lässt. Für einen Höhenruderausschlag von 0 $^\circ$ nimmt das Flugzeug einen Anstellwinkel von 2,76 $^\circ$ an. Ein Anstellwinkel von 0 $^\circ$ lässt sich bei einem $\eta$ von 1,13 $^\circ$ erreichen. Je negativer der Höhenruderausschlag ist, desto größer wird der Anstellwinkel.
  
\subsection{Auftriebsbeiwert über den Anstellwinkel}
Abbildung \ref{abb:do28_ca_alpha} zeigt den Auftriebsbeiwert $C_a$ über dem Anstellwinkel $\alpha$ der Do 28. Dieser lässt sich allgemein durch die Formel
\begin{equation}  \label{eq:theorie_ca}
C_A(\alpha) = C_{A\alpha}(\alpha - \alpha_0)
\end{equation}
beschreiben. Im Fall der Do 28 lautet die Formel
\begin{equation}  \label{eq:versuch_ca}
C_A(\alpha) = 0,11\alpha + 0,2
\end{equation}.
$\alpha_{0}$ beschreibt den Nullauftriebswinkel, den Winkel, bei dem das Flugzeug keinen Auftrieb mehr erzeugt. Dieser liegt bei der Do 28 bei $\alpha_{0}$ = -1,81 $^\circ$. $C_{A0}$ gibt den Auftriebsbeiwert an, welcher bei einem Anstellwinkel von $\alpha$ = 0 $\circ$ vorherrscht. Dieser beläuft sich bei der Do 28 auf einen Wert von $C_{A0}$ = 0,2. Die Regression verläuft für die Do 28 nur linear. In der Theorie verläuft der Auftrieb über den Anstellwinkel zunächst auch linear, bis ein Höhepunkt erreicht wird worauf der Auftrieb anschließend sinkt und kurz darauf aufgrund von ablösender Strömung zusammen bricht.  
Dieser maximale Auftriebsbereich fehlt aufgrund von nicht vorliegenden Messdaten im Grenzbereich des Flugzeugs. 
\subsection{LILIENTHAL-Polare}
\subsection{Widerstand über die Fluggeschwindigkeit}
\subsection{Staudruck und Fluggeschwindigkeit über dem Anstellwinkel}
\newpage

\section{Analyse durch Kilian Schultz}
\subsection{Höhenruder-Trimmkurve}
Im Plot Anstellwinkel über Höhenruderausschlag lässt sich eine leicht nach oben gewölbte Kurve erkennen, welche eine Korrelation zwischen nach unten gerichtetem Höhenruder und positivem Anstellwinkel erkennen lässt. Bei neutral eingestelltem Höhenruder lässt sich ein Anstellwinkel von etwa 2,5 Grad in positiver Richtung ablesen.\\
Auffällig sind die Werte des ersten Sinkfluges, hier bewirkt eine Änderung in der Stellung des Höhenruders deutlich größere Veränderungen am Anstellwinkel als in den Sinkflügen zwei bis vier.\\\\	

\subsection{Auftriebsbeiwert über den Anstellwinkel}
Der Auftriebsbeiwert über dem Anstellwinkel beschreibt mit Ausnahme des letzten Wertes des dritten Sinkfluges eine Gerade über welche eine Erhöhung des Anstellwinkels eine Erhöhung des Auftriebsbeiwertes zu erkennen ist.\\
Der Auftriebsbeiwert für einen neutralen Anstellwinkel lässt sich per linearer Regression über die Stichproben der Do-28 bei etwa 0,30 ablesen.\\
Der Nullauftriebsanstellwinkel $\alpha_{0}$ lässt sich ebenfalls per linearer Regression über die Stichproben bei etwa $-2,75$ Grad ablesen.\\\\
Der Auftriebsanstieg $C_{A\alpha}$ folgt aus dem Zusammenhang

\begin{equation*}
C_{A} = C_{A\alpha} * (\alpha-\alpha_{0})
\end{equation*}

\vspace{5mm}
\noindent und ist für $\alpha = 5$ Grad und $C_{A} = 0,75$ etwa $0,097$.\\\\

\subsection{LILIENTHAL-Polare}
Für die Lilienthal-Polare ergibt sich bei der Do-128 für den Wertebereich bis $C_{W} = 0,2$ ein steilerer Verlauf als bei der Do-28. Es werden also höhere Auftriebsbeiwerte bei niedrigeren Widerstandsbeiwerten erreicht als bei der Do-28.
Ab $C_{W} = 0,2$ liegen nur noch Daten für die Do-28 vor. Man erkennt, dass die letzten Werte des dritten Sinkfluges eine deutlichere Abnahme der Steigung in der Regressionskurve verursachen.\\
Die Gleichung der polynomialen Regression zweiten Grades erlaubt über einen Koeffizientenvergleich das Ablesen von Werten für $C_{W0}$ und $k$, sofern die Regression mit den Koeffizienten $1$ und $x^{2}$ durchgeführt wurde. Hierbei ist der alleinstehende skalare Wert $C_{W0}$ und der Koeffizient von $x^{2}$ ist $k$.

\begin{equation*}
C_{W} = C_{W0} + k C_{A}^{2} = a + b x^{2}
\end{equation*}

\vspace{5mm}
\noindent Zur weiteren Analyse ist die minimale reziproke Gleitzahl $\epsilon_{min}$ interessant. Durch Anlegen einer Tangente vom Ursprung an die Regressionskurve der Lilienthalpolare lassen sich ihre definierenden Werte $C_{A}^{*}$ und $C_{W}^{*}$ am Berührungspunkt ablesen. Diese Stelle markiert ebenfalls den Punkt des flachsten Gleitfluges, Sparflug genannt.\\
Alternativ kann bei bekanntem $C_{W0}$ und $k$, z.B. nach dem beschriebenen Koeffizientenvergleich, der Wert für $\epsilon_{min}$ über die folgenden Gleichungen hergeleitet werden:

\begin{equation*}
C_{W}^{*} = 2 C_{W0}
\end{equation*}

\begin{equation*}
C_{A}^{*} = \sqrt{\frac{C_{W0}}{k}}
\end{equation*}

\begin{equation*}
\epsilon_{min} = \frac{C_{W}^{*}}{C_{A}^{*}}
\end{equation*}

\vspace{5mm}
\noindent Weiterhin kann unter Zuhilfenahme der Flügelfläche, Gewichtskraft und Luftdichte die minimale Sinkgeschwindigkeit $w_{g_{min}}$ ermittelt werden.\\

\begin{equation*}
w_{g_{min}} = \sqrt{\frac{2mg}{\rho S}} \frac{4C_{W0}}{\left( \frac{3C_{W0}}{k}\right)^{\frac{3}{4}}}
\end{equation*}

\vspace{10mm}
\subsection{Widerstand über die Fluggeschwindigkeit}
Für die Do-128 kann man eine maßgebliche Korrelation zwischen dem Widerstand und der wahren Fluggeschwindigkeit erkennen. Dieser erhöht sich mit eben dieser in einem  nicht-linearen Verlauf.\\\\
Die Stichproben der Do-28 Messwerte geben ein leicht chaotisches Bild. Sinkflüge eins, zwei und drei folgen einem in sich selbst ähnlichen Verlauf wie die Do-128, Sinkflug 3 jedoch zeigt einen Verlauf in dem der Widerstand mit zunehmender Geschwindigkeit scheinbar zugenommen hat.\\
Außerdem ist zu vermerken, dass der Widerstand in Abhängigkeit der Fluggeschwindigkeit in Sinkflug 2 deutlich höher scheint als in der Sinkflügen eins und vier.\\\\

\subsection{Staudruck und Fluggeschwindigkeit über dem Anstellwinkel}
Anhand der Stichproben erkennt man, dass mit einer Erhöhung des Anstellwinkels sowohl Staudruck als auch wahre Fluggeschwindigkeit sinken. Der Betrag der Veränderung nimmt mit zunehmendem Anstellwinkel ab. Die Abnahme der Messwerte für Staudruck und Fluggeschwindigkeit bei Erhöhung des Anstellwinkels ist in etwa gleich.
\newpage

\section{Analyse durch Wentao Wu}
\subsection{Höhenruder-Trimmkurve}
\subsection{Auftriebsbeiwert über den Anstellwinkel}
\subsection{LILIENTHAL-Polare}
\subsection{Widerstand über die Fluggeschwindigkeit}
\subsection{Staudruck und Fluggeschwindigkeit über dem Anstellwinkel}
\newpage
	
