\chapter{Interpretation der Ergebnisse}
\label{chapter:interpretation}

\section{Analyse durch Marco De Gaetano}
\subsection{Höhenruder-Trimmkurve}
\subsection{Auftriebsbeiwert über den Anstellwinkel}
\subsection{LILIENTHAL-Polare}
\subsection{Widerstand über die Fluggeschwindigkeit}
\subsection{Staudruck und Fluggeschwindigkeit über dem Anstellwinkel}
\newpage

\section{Analyse durch Jens Karch}
\subsection{Höhenruder-Trimmkurve}
\subsection{Auftriebsbeiwert über den Anstellwinkel}
\subsection{LILIENTHAL-Polare}
\subsection{Widerstand über die Fluggeschwindigkeit}
\subsection{Staudruck und Fluggeschwindigkeit über dem Anstellwinkel}
\newpage

\section{Analyse durch Philip Margenfeld}
\subsection{Höhenruder-Trimmkurve}
\subsection{Auftriebsbeiwert über den Anstellwinkel}
\subsection{LILIENTHAL-Polare}
\subsection{Widerstand über die Fluggeschwindigkeit}
\subsection{Staudruck und Fluggeschwindigkeit über dem Anstellwinkel}
\newpage

\section{Analyse durch Kilian Schultz}
\subsection{Höhenruder-Trimmkurve}
Im Plot Anstellwinkel über Höhenruderausschlag lässt sich eine leicht nach oben gewölbte Kurve erkennen, welche eine Korrelation zwischen nach unten gerichtetem Höhenruder und positivem Anstellwinkel erkennen lässt. Bei neutral eingestelltem Höhenruder lässt sich ein Anstellwinkel von etwa 2,5 Grad in positiver Richtung ablesen.\\
Auffällig sind die Werte des ersten Sinkfluges, hier bewirkt eine Änderung in der Stellung des Höhenruders deutlich größere Veränderungen am Anstellwinkel als in den Sinkflügen zwei bis vier.\\\\	

\subsection{Auftriebsbeiwert über den Anstellwinkel}
Der Auftriebsbeiwert über dem Anstellwinkel beschreibt mit Ausnahme des letzten Wertes des dritten Sinkfluges eine Gerade über welche eine Erhöhung des Anstellwinkels eine Erhöhung des Auftriebsbeiwertes zu erkennen ist.\\
Der Auftriebsbeiwert für einen neutralen Anstellwinkel lässt sich per linearer Regression über die Stichproben der Do-28 bei etwa 0,30 ablesen.\\
Der Nullauftriebsanstellwinkel $\alpha_{0}$ lässt sich ebenfalls per linearer Regression über die Stichproben bei etwa $-2,75$ Grad ablesen.\\\\
Der Auftriebsanstieg $C_{A\alpha}$ folgt aus dem Zusammenhang

\begin{equation*}
C_{A} = C_{A\alpha} * (\alpha-\alpha_{0})
\end{equation*}

\vspace{5mm}
\noindent und ist für $\alpha = 5$ Grad und $C_{A} = 0,75$ etwa $0,097$.\\\\

\subsection{LILIENTHAL-Polare}
Für die Lilienthal-Polare ergibt sich bei der Do-128 für den Wertebereich bis $C_{W} = 0,2$ ein steilerer Verlauf als bei der Do-28. Es werden also höhere Auftriebsbeiwerte bei niedrigeren Widerstandsbeiwerten erreicht als bei der Do-28.
Ab $C_{W} = 0,2$ liegen nur noch Daten für die Do-28 vor. Man erkennt, dass die letzten Werte des dritten Sinkfluges eine deutlichere Abnahme der Steigung in der Regressionskurve verursachen.\\
Die Gleichung der polynomialen Regression zweiten Grades erlaubt über einen Koeffizientenvergleich das Ablesen von Werten für $C_{W0}$ und $k$, sofern die Regression mit den Koeffizienten $1$ und $x^{2}$ durchgeführt wurde. Hierbei ist der alleinstehende skalare Wert $C_{W0}$ und der Koeffizient von $x^{2}$ ist $k$.

\begin{equation*}
C_{W} = C_{W0} + k C_{A}^{2} = a + b x^{2}
\end{equation*}

\vspace{5mm}
\noindent Zur weiteren Analyse ist die minimale reziproke Gleitzahl $\epsilon_{min}$ interessant. Durch Anlegen einer Tangente vom Ursprung an die Regressionskurve der Lilienthalpolare lassen sich ihre definierenden Werte $C_{A}^{*}$ und $C_{W}^{*}$ am Berührungspunkt ablesen. Diese Stelle markiert ebenfalls den Punkt des flachsten Gleitfluges, Sparflug genannt.\\
Alternativ kann bei bekanntem $C_{W0}$ und $k$, z.B. nach dem beschriebenen Koeffizientenvergleich, der Wert für $\epsilon_{min}$ über die folgenden Gleichungen hergeleitet werden:

\begin{equation*}
C_{W}^{*} = 2 C_{W0}
\end{equation*}

\begin{equation*}
C_{A}^{*} = \sqrt{\frac{C_{W0}}{k}}
\end{equation*}

\begin{equation*}
\epsilon_{min} = \frac{C_{W}^{*}}{C_{A}^{*}}
\end{equation*}

\vspace{5mm}
\noindent Weiterhin kann unter Zuhilfenahme der Flügelfläche, Gewichtskraft und Luftdichte die minimale Sinkgeschwindigkeit $w_{g_{min}}$ ermittelt werden.\\

\begin{equation*}
w_{g_{min}} = \sqrt{\frac{2mg}{\rho S}} \frac{4C_{W0}}{\left( \frac{3C_{W0}}{k}\right)^{\frac{3}{4}}}
\end{equation*}

\vspace{10mm}
\subsection{Widerstand über die Fluggeschwindigkeit}
Für die Do-128 kann man eine maßgebliche Korrelation zwischen dem Widerstand und der wahren Fluggeschwindigkeit erkennen. Dieser erhöht sich mit eben dieser in einem  nicht-linearen Verlauf.\\\\
Die Stichproben der Do-28 Messwerte geben ein leicht chaotisches Bild. Sinkflüge eins, zwei und drei folgen einem in sich selbst ähnlichen Verlauf wie die Do-128, Sinkflug 3 jedoch zeigt einen Verlauf in dem der Widerstand mit zunehmender Geschwindigkeit scheinbar zugenommen hat.\\
Außerdem ist zu vermerken, dass der Widerstand in Abhängigkeit der Fluggeschwindigkeit in Sinkflug 2 deutlich höher scheint als in der Sinkflügen eins und vier.\\\\

\subsection{Staudruck und Fluggeschwindigkeit über dem Anstellwinkel}
Anhand der Stichproben erkennt man, dass mit einer Erhöhung des Anstellwinkels sowohl Staudruck als auch wahre Fluggeschwindigkeit sinken. Der Betrag der Veränderung nimmt mit zunehmendem Anstellwinkel ab. Die Abnahme der Messwerte für Staudruck und Fluggeschwindigkeit bei Erhöhung des Anstellwinkels ist in etwa gleich.
\newpage

\section{Analyse durch Wentao Wu}
\subsection{Höhenruder-Trimmkurve}
\subsection{Auftriebsbeiwert über den Anstellwinkel}
\subsection{LILIENTHAL-Polare}
\subsection{Widerstand über die Fluggeschwindigkeit}
\subsection{Staudruck und Fluggeschwindigkeit über dem Anstellwinkel}
\newpage
	
