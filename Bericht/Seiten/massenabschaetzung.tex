\chapter{Massenabschätzung}
\label{chapter:masse}
Die Massenabschätzung der Do 28 erfolgt auf der Grundlage von zuvor ermittelten Messdaten, welche uns in diesem Labor zur Verfügung gestellt werden. \\
Wie in Gleichung 2.X zu erkennen ist, setzt sich die Gesamtmasse des Flugzeugs aus der Rüstmasse, der Besatzungsmasse und der aktuellen Treibstoffmasse zusammen. Die Rüst- und Besatzungsmasse bleiben über den Flug konstant, lediglich die Kraftstoffmasse verringert sich auf Grund des Verbrauchs. Wir wissen, dass der Tank bei Versuchsstart zu 70 \% und bei Versuchsende zu 50 \% gefüllt ist. Das maximale Tankvolumen entspricht 822 {l} und die verwendete Kraftstoffdichte liegt bei $\rho_{Kraftst}$ = 0,72 {kg/l}. Damit lässt sich mit Gleichung 2.Y die maximale Kraftstoffmasse berechnen und mit den gegebenen Volumina zu Versuchsstart und -ende auf die dort vorliegende Kraftstoffmasse schließen.



\newpage