\chapter{Massenabschätzung}
\label{chapter:masse}
Die Massenabschätzung der Do 28 erfolgt auf der Grundlage von zuvor ermittelten Messdaten, welche uns in diesem Labor zur Verfügung gestellt werden. \\

\begin{table}[h]
	\centering
	\begin{tabular}{|l|c|r|}
		\hline
	\textbf{Größe}		 & \textbf{Wert}& \textbf{Einheit} \\ \hline
	$m_{Rüst}$  		 & 2936			& {kg}        \\ \hline
	$m_{Besatzung}$ 	 & 346			& {kg}		  \\ \hline
	$V_{Kraftst,max}$    & 822			& {l}		  \\ \hline
	$\rho_{Kraftst}$ 	 & 0,72			& {kg/l}	  \\ \hline
	$m_{Kraftst,Start}$	 & 414			& {kg}		  \\ \hline
	$m_{Kraftst,Ende}$	 & 296			& {kg}		  \\ \hline	


\end{tabular}
	\caption{Massenaufstellung}
\end{table}

Wie in Gleichung 2.X zu erkennen ist, setzt sich die Gesamtmasse des Flugzeugs aus der Rüstmasse, der Besatzungsmasse und der aktuellen Treibstoffmasse zusammen. Die Rüst- und Besatzungsmasse bleiben über den Flug konstant, lediglich die Kraftstoffmasse verringert sich auf Grund des Verbrauchs. Wir wissen, dass der Tank bei Versuchsstart zu 70 \% und bei Versuchsende zu 50 \% gefüllt ist. Das maximale Tankvolumen entspricht 822 {l} und die verwendete Kraftstoffdichte liegt bei $\rho_{Kraftst}$ = 0,72 {kg/l}. Damit lässt sich mit Gleichung 2.Y die maximale Kraftstoffmasse berechnen und mit den gegebenen Volumina zu Versuchsstart und -ende auf die dort vorliegende Kraftstoffmasse schließen.



\newpage