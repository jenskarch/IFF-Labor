\chapter{Versuchsdurchführung}
\label{chapter:versuch}
Der Flugversuch fand am  Dienstag den 21.05.2019 um 13:00 Uhr am Forschungsflughafen Braunschweig-Wolfsburg (EDVE) statt. Insgesamt waren sechs Personen an Bord des Forschungsflugzeuges, darunter ein Pilot vom Institut für Flugführung sowie wir fünf Laboranten. Damit ergab sich ein Besatzungsgewicht von 427 $kg$.\\
Das genutzte Forschungsflugzeug war eine Do 128-6 aus dem Hause Dornier Aircraft mit der Registrierung D-IBUF. Die technischen Daten dieses Flugzeugmusters können aus nachfolgender Tabelle entnommen werden:\\

\begin{table}[h]
	\centering
	\begin{tabular}{|l|c|r|}
		\hline
		\textbf{Bezeichnung}         	& \textbf{Formelzeichen} & \textbf{Wert}           		\\ \hline
		Flügelbezugsfläche           	& S                      & 29 $m^2$ 					\\ \hline
		Spannweite                   	& b                      & 15,55 $m$                 	\\ \hline
		Rüstmasse                    	& $m_{Ruest}$             & 3188 $kg$                 	\\ \hline
		Kraftstoffvolumen max. 			& $V_{Kraftst,max}$      & 1470 $l$            			\\ \hline
		Kraftstoffdichte             	& $\rho_{Kraftst}$       & 0,784 $\frac{kg}{l}$      	\\ \hline
		%Kraftstoffmasse max.		 	& $m_{Kraftst,max}$      & 1152,5 $kg$            		\\ \hline
		%Kraftstoffmasse vor Versuch 	& $m_{Kraftst}$			 & 767,5 $kg$) 					\\ \hline
	\end{tabular}
	\caption{Technische Daten des Versuchsflugzeugs Do 128-6}
\end{table}
\vspace{0.3cm}

\noindent Zum Zeitpunkt des Flugversuchs war die Wolkenuntergrenze etwas niedrig (Wolkenuntergrenze bei 1600 $ft$ AMSL laut METAR, entspricht etwa 700 $ft$ AGL), doch im Verlauf des Flugversuchs stieg die Wolkenuntergrenze an, sodass der Flugversuch problemlos durchgeführt werden konnte. Die Bodenwerte während des Versuchs lagen bei einer Temperatur von 20 $^\circ C$ (rund 293 $K$) und einem Druckwert von 1006 $hPa$. Die lokale METAR Meldung kurz vor dem Versuch lautete wie folgt:\\\\

\vspace{-0.3cm}
\begin{center}
\framebox[0.7\textwidth][c]{EDVE 211120Z 31010KT 280V350 9999 SCT016 20/15 Q1006}
\end{center}
\vspace{0.8cm}

\noindent Um 13:18 Uhr hob das Flugzeug auf Piste 26 in Richtung Westen ab. Nach etwa 8 Minuten begann der eigentliche Versuch, indem der Pilot vier stationäre Sinkflüge mit unterschiedlichen Geschwindigkeiten einleitete. Ziel war es dabei einen Gleitflug zu simulieren, also die Sinkflüge ohne Schub durchzuführen. Da die Triebwerke allerdings Widerstand erzeugen, wurde der Schub nicht vollständig zurückgefahren, sondern nur soweit, dass der Triebwerkswiderstand überwunden wird. Durch diese Methode lassen sich die Luftkräfte (Auftrieb und Widerstand) nach Gleichung \ref{eq:kräftegleichgewichtAuftrieb} und \ref{eq:kräftegleichgewichtWiderstand} berechnen.\\\\  
Um 13:40 Uhr landete das Flugzeug wieder am Forschungsflugzeug. Der Versuch war damit beendet.

\section{Messdaten}
Alle Sinkflüge wurden in einem Höheninterval von 1000 ft durchgeführt. Dabei starteten wir bei etwa \textbf{2500 ft} und sanken auf rund \textbf{1500 ft} ab. Sämtliche Höhenangaben beziehen sich dabei auf die \textbf{angezeigte Höhe über der Bezugsfläche 1013,25 hPa}.\\
Die Geschwindigkeit wurde als Stellfaktor der vier verschiedenen Sinkflüge gewählt. Beim ersten Sinkflug starteten wir mit einer angezeigten Geschwindigkeit von 80 kt, wobei versucht wurde diese Geschwindigkeit möglichst konstant bis zum Ende des Sinkflugs zu halten. Der zweite Sinkflug wurde dann bei 100 kt, der dritte bei 120 kt und der letzte bei 140 kt angezeigter Geschwindigkeit durchgeführt.\\
Zu Beginn und Ende jedes Sinkflugs wurden die Temperatur, sowie der aktuelle (seit Triebwerkstart) verbrauchte Kraftstoff abgelesen. Zudem wurde die Zeit gestoppt, die für die Sinkflüge über 1000 ft benötigt wurden.\\\\
Die nachfolgende Tabelle zeigt unsere ermittelten Messdaten zu den Sinkflügen. Die Orginalfassung unserer Protokolle kann zusätzlich auch im Anhang gefunden werden.\\

\begin{table}[h]
	\begin{tabular}{|c|c|c|c|c|c|c|}
		\hline
		& \multicolumn{1}{l|}{\textbf{$V_{IAS}$}} & \multicolumn{1}{l|}{\textbf{Zeitintervall}} & \multicolumn{1}{l|}{\textbf{Temp. (Start)}} & \multicolumn{1}{l|}{\textbf{Temp. (Ende)}} & \multicolumn{1}{l|}{\textbf{Verbrauch (Start)}} & \multicolumn{1}{l|}{\textbf{Verbrauch (Ende)}} \\ \hline
		1 & 80 kt          & 1 min 38 s         & 12,0 °C         & 14,5 °C       & 83 lbs / 37,65 kg        & 90 lbs / 40,82 kg    \\ \hline
		2 & 100 kt         & 1 min 08 s         & 12,0 °C         & 15,0 °C       & 107 lbs / 48,53 kg       & 113 lbs / 51,26 kg   \\ \hline
		3 & 120 kt         & 48 s               & 13,0 °C         & 16,0 °C       & 130 lbs / 58,97 kg       & 134 lbs / 60,78 kg   \\ \hline
		4 & 140 kt         & 31 s               & 13,0  °C        & 15,0 °C       & 148 lbs / 67,13 kg       & 150 lbs / 68,04 kg   \\ \hline
	\end{tabular}
	\caption{Messergebnisse aus dem Flugversuch} \label{tab:messergebnisse}
\end{table}

\noindent Für die Umrechnung der in Pfund abgelesenen Verbrauchswerte in das Kilogramm, wurde dabei folgende Umrechnungsformel zugrunde gelegt.\\

\begin{equation}
m_{kg} = \frac{m_{lbs}}{2,205}
\end{equation}

\newpage
