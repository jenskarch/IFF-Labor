\chapter{Versuchsdurchführung}
\label{chapter:versuch}
Am Tag 21.05.2019 um 13:00 Uhr in Braunschweig Flughafen wurde der Versuch durchgeführt. Mit Pilot flogen 6 Personen, die sich 427 $kg$ als die Besatzungsgewicht besitzen, zusammen. In Prinzipiell war die damalige Wetter nicht typisch geeignet für Flug, weil es  sanft windig und schwer bewölkt war. Die Temperatur war $15^\circ\text{C}$ und der Bodendruck war 1006 $hPa$. \\
Der angewendete Flugzeug zum Versuch ist Do 128, der sich zu einem kleinen Flugzeug eignet. Die Spannbereite $b$ vom Do 128 ist 15,00 $m$ und die Spannfläche beträgt 29 ${m}^{2}$. Do 128 ist gar nicht so schwer bzw. die Rüstmasse ist nur 3188 $kg$. Dieser Flugzeug besitzt zwei Haupttanks, die jeweils 440 $lbs$ (199,58 $kg$) wiegen, und zwei Reservetanks, die jeweils 406 $lbs$ (184.16 $kg$) wiegen, und trägt maximal $V_{Kraftst,max}$ 1470 $l$ Kraftstoff, der mit Dichte $rho_{Kraftst}$ 0.784 $kg/l$ ist. \\
Es werden vier stationäre Sinkflüge bei unterschiedlichen Fluggeschwindigkeiten durchgeführt. Da  es im stationäre Sinkflug keinen Schub gibt und Kraftgleichgewicht herrscht, braucht man $A$ und $W$ auszurechnen statt direkt zu messen. \\
Um 13:18 flog der Flugzeug ab bzw. Take- off. Nach ungefähr 8 Minuten fing der erste Versuch an. In dem ersten Versuch stieg der Flugzeug in 5 Minuten in der Höhe 2500 $ft$ (609 $m$), sank mit Geschwindigkeit 80 $kts$ (41,16 $m/s$) in die Höhe 1450 $ft$ (441,96 $m$). Die Außentemperatur stieg sich von $12^\circ\text{C}$ bis $14.5^\circ\text{C}$ auf. Nach kurzer Zeit fing der Flugzeug dem Sinkflug von 2500 $ft$ (762 $m$) mit Geschwindigkeit 100 $kts$ (51,44 $m/s$) bis 2000 $ft$ (609,6 $m$). Inzwischen stiegt die Außentemperatur von $12^\circ\text{C}$ bis $15^\circ\text{C}$ auf. In den weiteren zwei Versuchen waren die Anfangshöhe und Endhöhe immer gleich wie bei zweitem Versuch bzw. 2500 $ft$ (762 $m$) und 2000 $ft$ (609,6 $m$) und der dritte Versuch war mit Geschwindigkeit 120 $kts$ (61,73 $m/s$) und der vierte Versuch war mit Geschwindigkeit 140 $kts$ (72,02 $m/s$). Die Außentemperatur im dritten Versuch stieg von $13^\circ\text{C}$ bis $16^\circ\text{C}$ und im viertem Versuch von $13^\circ\text{C}$ bis $15^\circ\text{C}$ auf. Die vier Versuchen dauerten jeweils 1 Minuten 38 Sekunden, 1 Minuten 8 Sekunden, 48 Sekunden und 31 Sekunden. Um 13:40 Uhr landet der Flugzeug wieder an Braunschweig Flughafen.

\newpage
